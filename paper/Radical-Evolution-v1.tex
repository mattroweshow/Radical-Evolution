% THIS IS SIGPROC-SP.TEX - VERSION 3.1
% WORKS WITH V3.2SP OF ACM_PROC_ARTICLE-SP.CLS
% APRIL 2009
%
% It is an example file showing how to use the 'acm_proc_article-sp.cls' V3.2SP
% LaTeX2e document class file for Conference Proceedings submissions.
% ----------------------------------------------------------------------------------------------------------------
% This .tex file (and associated .cls V3.2SP) *DOES NOT* produce:
%       1) The Permission Statement
%       2) The Conference (location) Info information
%       3) The Copyright Line with ACM data
%       4) Page numbering
% ---------------------------------------------------------------------------------------------------------------
% It is an example which *does* use the .bib file (from which the .bbl file
% is produced).
% REMEMBER HOWEVER: After having produced the .bbl file,
% and prior to final submission,
% you need to 'insert'  your .bbl file into your source .tex file so as to provide
% ONE 'self-contained' source file.
%
% Questions regarding SIGS should be sent to
% Adrienne Griscti ---> griscti@acm.org
%
% Questions/suggestions regarding the guidelines, .tex and .cls files, etc. to
% Gerald Murray ---> murray@hq.acm.org
%
% For tracking purposes - this is V3.1SP - APRIL 2009

\documentclass{acm_proc_article-sp}

\begin{document}

\title{Mining Radicalisation Trajectories from Social Media Discourse}

\numberofauthors{2} %  in this sample file, there are a *total*
% of EIGHT authors. SIX appear on the 'first-page' (for formatting
% reasons) and the remaining two appear in the \additionalauthors section.
%
\author{
% You can go ahead and credit any number of authors here,
% e.g. one 'row of three' or two rows (consisting of one row of three
% and a second row of one, two or three).
%
% The command \alignauthor (no curly braces needed) should
% precede each author name, affiliation/snail-mail address and
% e-mail address. Additionally, tag each line of
% affiliation/address with \affaddr, and tag the
% e-mail address with \email.
%
% 1st. author
\alignauthor
Hassan Saif\\
       \affaddr{Knowledge Media Institute}\\
       \affaddr{The Open University}\\
       \affaddr{Milton Keynes, UK}\\
       \email{hassan.saif@open.ac.uk}
% 2nd. author
\alignauthor
Matthew Rowe\\
       \affaddr{School of Computing and Communications}\\
       \affaddr{Lancaster University}\\
       \affaddr{Lancaster, UK}\\
       \email{m.rowe@lancaster.ac.uk}
}
% There's nothing stopping you putting the seventh, eighth, etc.
% author on the opening page (as the 'third row') but we ask,
% for aesthetic reasons that you place these 'additional authors'
% in the \additional authors block, viz.

\maketitle
\begin{abstract}
to do
\end{abstract}

% A category with the (minimum) three required fields
\category{H.4}{Information Systems Applications}{Miscellaneous}
%A category including the fourth, optional field follows...
\category{D.2.8}{Software Engineering}{Metrics}[complexity measures, performance measures]

\terms{Theory}

\keywords{ACM proceedings, \LaTeX, text tagging} % NOT required for Proceedings

\section{Introduction}
The aim of this study is to understand the paths that social media users exhibit, in their discourse and behaviour, on the way to discussing and sharing radicalised social media content.

Assumptions:\\
1. That `sharing' of radicalised/fundamental content is a \emph{signifier} of radicalisation.\\

\subsection{Research Questions}
\begin{enumerate}
	\item \textbf{RQ1:} Can we identify `catalysts' of radicalisation? And are they evident within the data?
	\item \textbf{RQ2:} What influences people to change their discourse? And what does `change' actually look like?
	\item \textbf{RQ3:} [Seriously draconian, so not sure about this one] How can we forecast radicalisation? 
\end{enumerate}


\subsection{Steps}
In order to conduct this research we will carry out the following research steps:

\begin{enumerate}
	\item \emph{Data Gathering and Validation:} Use the data provided by D. Greene with the 652 user ids to first validate that those users exist and to also examine how recent their data is.
	Then gather the followers of those \emph{source} (652) users and pick out the ones based in the UK.
	Repeat these steps to gather a sufficient number of people to analyse - i.e. going 2-hops away.
	
	\item \emph{Radicalisation Mining:} Create techniques to detect when someone has shared/posted radical content.
	At first we can just track when this occurs and what the content is that those users are sharing.
	
	\item \emph{Catalyst Detection:} Once we have developed the approach to pick out when someone is sharing radical content, then we can identify the points at which they begin this (i.e. changepoint analysis techniques could be used here).
	Once we have that then we examine the conditions under which users tend to change - we can also pick out how such conditions provide the setting for users to be influenced in what they share
\end{enumerate}

I think that we can probably stop the work at this point and we will have enough, without having to do the forecasting of radicalisation.



%
% The following two commands are all you need in the
% initial runs of your .tex file to
% produce the bibliography for the citations in your paper.
\bibliographystyle{abbrv}
\bibliography{sigproc}  % sigproc.bib is the name of the Bibliography in this case


\balancecolumns
% That's all folks!
\end{document}
